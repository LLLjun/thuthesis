% !TeX root = ../main.tex

% 中英文摘要和关键字

\begin{abstract}
  推荐系统目前被广泛使用到各类场景,例如电商、社交、广告等等。其中推荐系统的核心组件之一是近似近邻搜索算法,该算法的目的是从百万上亿的数据中检索与给定查询最相似的结果。其中基于图结构的近似近邻搜索算法在多个公开基准上的测试都优于其他类型的算法,而被广泛适用于推荐系统中,作为推荐系统的核心组件之一。本文以推荐系统作为近似近邻搜索的典型场景,开展面向\textbf{图结构推荐算法}和硬件架构的软硬件协同设计研究。随着大数据时代的快速到来,推荐系统所需要处理的数据量也急剧增长,图结构推荐算法在算法和硬件层面也都面临新的问题和挑战。因此,本文以推荐系统作为近似近邻搜索算法的典型场景,从算法、架构和系统三个层面开展面向图结构推荐算法和硬件架构的软硬件协同设计研究。

  在算法层面,在各类近似近邻搜索算法中,尽管图结构推荐算法拥有出色的搜索性能,但是面临构建开销大和搜索时存在冗余的问题。本文针对这两个问题分别提出了构建时优化和搜算阶段的分析,指出冗余的原因是由于图结构中不同区域的连接关系不同所导致。

  在架构层面,图结构推荐算法在现有硬件架构上(如CPU和GPU)面临执行能效低下的问题。其核心原因是图结构引入大量的间接索引问题,导致现有的冯诺依曼这种存算分离的架构在数据传输的开销很大。因此本文提出了一种DRAM-SSD的层次化近存储处理架构,是一个针对图结构推荐算法所设计的高效硬件处理架构。

  在系统层面,随着数据规模的增加,拓展性是一个必须要考虑的问题。由于节点间通信的昂贵开销,现有多机方案都采用节点间无通信的方案,但是这种方案受到图结构的影响导致很差的拓展性。因此,本文在系统层次进一步提出了可拓展的处理方案,并在FPGA上进行了功能验证,取得了超过近存储架构设计的性能。

  % 关键词用“英文逗号”分隔,输出时会自动处理为正确的分隔符
  \thusetup{
    keywords = {推荐系统, 近似近邻搜索, 软硬件协同设计, 近存储计算},
  }
\end{abstract}

\begin{abstract*}
  TODO

  % Use comma as separator when inputting
  \thusetup{
    keywords* = {Recommendation system, approximate nearest neighbor search, software-hardware co-design, near-memory computing},
  }
\end{abstract*}
